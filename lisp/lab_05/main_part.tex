\chapter{Практические задания}

Используя функционалы:

\section{Задание №1}

Напишите функцию, которая уменьшает на 10 все числа из списка-аргумента этой функции, проходя по верхнему уровню списковых ячеек.  (* Список смешанный структурированный)

\begin{lstlisting}[language=Lisp]
	(defun minus10 (lst)
		(mapcar #'(lambda (x)
			(cond ((numberp x) (- x 10))
				((atom x) x)
				(T (minus10 x)))) lst))
\end{lstlisting}

\section{Задание №2}

Написать функцию которая получает как аргумент список чисел, а возвращает список квадратов этих чисел в том же порядке.

\begin{lstlisting}[language=Lisp]
	(defun squares (lst)
		(mapcar #'(lambda (x) (* x x)) lst))
\end{lstlisting}

\section{Задание №3}

Напишите функцию, которая умножает на заданное число-аргумент все числа из заданного списка-аргумента, когда

a) все элементы списка --- числа,

\begin{lstlisting}[language=Lisp]
(defun mul-num (lst num)
	(mapcar #'(lambda (x) (* x num)) lst))
\end{lstlisting}

б) элементы списка --- любые объекты.

\begin{lstlisting}[language=Lisp]
(defun mul-num (lst num)
	(mapcar #'(lambda (x)
		(cond ((numberp x) (* x num))
			((atom x) x)
			(T (mul-num x num)))) lst))
\end{lstlisting}

\section{Задание №4}

Написать функцию, которая по своему списку-аргументу lst определяет является ли он палиндромом (то есть равны ли lst и (reverse lst)), для одноуровнего смешанного списка.

\begin{lstlisting}[language=Lisp]
	(defun is-palindrome (lst)
		(equal lst (reverse lst)))
\end{lstlisting}

\section{Задание №5}

Используя функционалы, написать предикат set-equal, который возвращает t, если два его множества-аргумента (одноуровневые списки) содержат одни и те же элементы, порядок которых не имеет значения.

\begin{lstlisting}[language=Lisp]
	(defun is-in (el st)
		(reduce #'(lambda (x y) (or x y))
			(mapcar #'(lambda (x) (equal x el)) st) :initial-value Nil))
	
	(defun is-subset (st1 st2)
		(reduce #'(lambda (x y) (and x y))
			(mapcar #'(lambda (x) (is-in x st2)) st1) :initial-value T))
	
	(defun set-equal (st1 st2)
		(and (is-subset st1 st2) (is-subset st2 st1)))
\end{lstlisting}

\section{Задание №6}

Напишите функцию, select-between, которая из списка-аргумента, содержащего только числа, выбирает только те, которые расположены между двумя указанными числами --- границами-аргументами и возвращает их в виде списка (упорядоченного по возрастанию (+ 2 балла)).

\begin{lstlisting}[language=Lisp]
	(defun select-between (lst begin end)
		(sort (remove-if-not #'(lambda (x) (< begin x end)) lst) '<))
\end{lstlisting}

\section{Задание №7}

Написать функцию, вычисляющую декартово произведение двух своих списковаргументов. ( Напомним, что А х В это множество всевозможных пар (a b), где а принадлежит А, принадлежит В.)

\begin{lstlisting}[language=Lisp]
	(defun cartesian (lst1 lst2)
		(mapcan #'(lambda (el1)
			(mapcar #'(lambda (el2) (list el1 el2)) lst2)) lst1))
\end{lstlisting}

\section{Задание №8}

Почему так реализовано reduce, в чем причина?
\begin{lstlisting}[language=Lisp]
	(reduce #'+ ()) -> 0
	(reduce #'* ()) -> 1
\end{lstlisting}

Функционал reduce выполняет следующее преобразование исходного списка L=>(e1 e2 … en) с использованием начального значения A и бинарной операции-функции F: (reduce F L A) = (F(…(F(F A e1) e2))…en). Можно сказать, что значение A логически добавляется в начало списка L.

\begin{itemize}
	\item Если список содержит ровно один элемент и начальное значение не задано, то этот элемент возвращается, а функция не вызывается. 
	\item Если список пуст и задано начальное значение, то возвращается начальное значение, а функция не вызывается. 
	\item Если список пуст и начальное значение не задано, то функция вызывается без аргументов, и reduce возвращает то, что вернет функция. Это единственный случай, когда функция вызывается с другим количеством аргументов, кроме двух.
\end{itemize}

(reduce \#'+ ()): список пуст, начальное значение не указано, следовательно, функция + вызывается без аргументов, и reduce возвращает то, что вернет функция +. (+) => 0, поэтому и (reduce \#'+ ()) => 0.

С умножением аналогично.

\section{Задание №9}

* Пусть list-of-list список, состоящий из списков. Написать функцию, которая
вычисляет сумму длин всех элементов list-of-list (количество атомов), т.е. например
для аргумента
((1 2) (3 4)) -> 4.

\begin{lstlisting}[language=Lisp]
	(defun get-length (lst)
		(reduce #'+ (mapcar #'(lambda (x) (cond ((atom x) 1)
												(T (get-length x)))) lst)))
\end{lstlisting}
