\chapter{Практические задания}

\section{Задание №1}

Чем принципиально отличаются функции cons, list, append?

Пусть

\begin{lstlisting}[language=Lisp]
	(setf lst1 '( a b c))
	(setf lst2 '( d e))
\end{lstlisting}

Каковы результаты вычисления следующих выражений?

\begin{lstlisting}[language=Lisp]
	(cons lstl lst2)  => ((a b c) d e)
	(list lst1 lst2)  => ((a b c) ( d e))
	(append lst1 lst2)  => (a b c d e)
\end{lstlisting}

\section{Задание №2}

Каковы результаты вычисления следующих выражений, и почему?

\begin{lstlisting}[language=Lisp]
	(reverse '(a b c))  => (c b a)
	(reverse '(a b (c (d)))) => ((c (d)) b a)
	(reverse '(a)) => (a)
	(last '(a b c)) => (c)
	(last '(a)) => (a)
	(last '((a b c))) => ((a b c))
	(reverse ()) => Nil
	(reverse '((a b c))) => ((a b c))
	(last '(a b (c))) => ((c))
	(last ()) => Nil
\end{lstlisting}

\section{Задание №3}

Написать, по крайней мере, два варианта функции, которая возвращает
последний элемент своего списка-аргумента

\begin{lstlisting}[language=Lisp]
	(defun get-last1 (lst)
		(car (last lst))
		
	(defun get-last2 (lst)
		(if (cdr lst)
			(get-last2 (cdr lst))
			(car lst)))
			
	(defun get-last3 (lst)
		(car (reverse lst)))
\end{lstlisting}

\section{Задание №4}

Написать, по крайней мере, два варианта функции, которая возвращает
свой список аргумент без последнего элемента.

\begin{lstlisting}[language=Lisp]
	(defun get-no-last1 (lst)
		(reverse (cdr (reverse lst))))
		
	(defun get-no-last2 (lst)
		(if (cdr lst)
			(cons (car lst) (get-no-last2 (cdr lst)))))
\end{lstlisting}

\section{Задание №5}

Напишите функцию swap-first-last, которая переставляет в списке-аргументе первый и последний элементы.

\begin{lstlisting}[language=Lisp]
	(defun swap-first-last (lst)
		(cond ((null (cdr lst)) lst)
			(T (cons (car (last lst))
				(reverse (cons (car lst) (cdr (reverse (cdr lst)))))))))
\end{lstlisting}

\section{Задание №6}

Написать простой вариант игры в кости, в котором бросаются две
правильные кости. Если сумма выпавших очков равна 7 или 11 —
выигрыш, если выпало (1,1) или (6,6) — игрок имеет право снова
бросить кости, во всех остальных случаях ход переходит ко второму
игроку, но запоминается сумма выпавших очков. Если второй игрок не
выигрывает абсолютно, то выигрывает тот игрок, у которого больше
очков. Результат игры и значения выпавших костей выводить на экран с
помощью функции print.

\begin{lstlisting}[language=Lisp]
	(setf *random-state* (make-random-state T))

	(defun roll-dice ()
		(list (+ (random 6) 1) (+ (random 6) 1)))

	(defun dice-sum (dice)
		(apply #'+ dice))

	(defun check-win (sum)
		(or (= sum 7) (= sum 11)))

	(defun check-repeat (dice)
		(let ((f (car dice))
			(s (cadr dice)))
			(or (= f s 1) (= f s 6))))

	(defun player-round (who)
		(let* ((player-dice (roll-dice))
			   (player-sum (dice-sum player-dice)))
			  (print who)
			  (print player-dice)
			  (cond ((check-win player-sum) Nil)
					((check-repeat player-dice) (player-round who))
					(T player-sum))))

	(defun dice ()
		(let ((first-sum (player-round "first")))
			(if (not first-sum)
				(print "First player is winner!")
				(let ((second-sum (player-round "second")))
					(cond ((or (not second-sum) (< first-sum second-sum))
							  (print "Second player is winner!"))
						  ((> first-sum second-sum)
							  (print "First player is winner!"))
						  (T (print "Draw.")))))))

(dice)
\end{lstlisting}

\section{Задание №7}

Написать функцию, которая по своему списку-аргументу lst определяет
является ли он палиндромом (то есть равны ли lst и (reverse lst)).

\begin{lstlisting}[language=Lisp]
	(defun is-palindrome (lst)
		(equal lst (reverse lst)))
\end{lstlisting}

\section{Задание №8}

Напишите свои необходимые функции, которые обрабатывают таблицу из
4-х точечных пар:
(страна . столица), и возвращают по стране - столицу, а по столице —
страну.

\begin{lstlisting}[language=Lisp]
	(defun get-capital (table country)
		(cond ((null table) Nil)
			((equal (caar table) country) (cdar table))
			(T (get-capital (cdr table) country))))
			
	(defun get-country (table capital)
		(cond ((null table) Nil)
			((equal (cdar table) capital) (caar table))
			(T (get-country (cdr table) capital))))
\end{lstlisting}

\section{Задание №9}

Напишите функцию, которая умножает на заданное число-аргумент
первый числовой элемент списка из заданного 3-х элементного списка-аргумента, когда
a) все элементы списка --- числа,
б) элементы списка -- любые объекты.

a) все элементы списка --- числа
\begin{lstlisting}[language=Lisp]
	(defun mul-first-nums (lst num)
		(and lst (cons (* (car lst) num) (cdr lst))))
\end{lstlisting}

б) элементы списка -- любые объекты
\begin{lstlisting}[language=Lisp]
	(defun mul-first-num (lst num)
		(cond ((null lst) lst)
			((numberp (car lst)) (cons (* (car lst) num) (cdr lst)))
			(T (cons (car lst) (mul-first-num (cdr lst) num)))))
\end{lstlisting}
