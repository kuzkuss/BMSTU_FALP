\chapter{Практические задания}

\section{Задание №1}

Написать функцию, которая принимает целое число и возвращает первое
четное число, не меньшее аргумента.

\begin{lstlisting}[language=Lisp]
	(defun first-greater-even (num)
		(if (evenp num) num (+ num 1)))
\end{lstlisting}

\section{Задание №2}

Написать функцию, которая принимает число и возвращает число
того же знака, но с модулем на 1 больше модуля аргумента.

\begin{lstlisting}[language=Lisp]
	(defun module-plus (num)
		(+ num (if (> num 0) 1 -1)))
\end{lstlisting}

\section{Задание №3}

Написать функцию, которая принимает два числа и возвращает
список из этих чисел, расположенный по возрастанию.

\begin{lstlisting}[language=Lisp]
	(defun make-growing-list (a b)
		(if (< a b) (list a b) (list b a)))
\end{lstlisting}

\section{Задание №4}

Написать функцию, которая принимает три числа и возвращает Т только
тогда, когда первое число расположено между вторым и третьим.

\begin{lstlisting}[language=Lisp]
	(defun is-between (a b c)
		(or (and (> a b) (< a c)) (and (> a c) (< a b))))
\end{lstlisting}

\section{Задание №5}

Каков результат вычисления следующих выражений?

\begin{lstlisting}[language=Lisp]
	(and 'fee 'fie 'foe)  => foe
	(or nil 'fie 'foe) => fie
	(and (equal 'abc 'abc) 'yes) => yes
	(or 'fee 'fie 'foe) => fee
	(and nil 'fie 'foe) => nil
	(or (equal 'abc 'abc) 'yes) => T
\end{lstlisting}

\section{Задание №6}

Написать предикат, который принимает два числа-аргумента и возвращает
Т, если первое число не меньше второго.

\begin{lstlisting}[language=Lisp]
	(defun greaterp (a b)
		(>= a b))
\end{lstlisting}

\section{Задание №7}

Какой из следующих двух вариантов предиката ошибочен и почему?

\begin{lstlisting}[language=Lisp]
	(defun pred1 (x)
		(and (numberp x) (plusp x)))

	(defun pred2 (x)
		(and (plusp x) (numberp x)))
\end{lstlisting}

Ошибочным является второй вариант, так как функция plusp принимает на
вход один аргумент типа number, из-за чего аргументы, не являющиеся числами,
будут вызывать ошибку. При этом в первом варианте, если при проверке,
является ли аргумент числом, получится значение Nil, то and вернет его в
качестве результата, не продолжая дальнейшие вычисления, а при передачи
числа будет проведена проверка на положительность.

\section{Задание №8}

Решить задачу 4, используя для ее решения конструкции:
только IF, только COND, только AND/OR.

\begin{lstlisting}[language=Lisp]
	;; if
	(defun is-between-if (a b c)
		(if (> a b)
			(< a c)
			
			(if (< a b)
				(> a c))))

	;; cond
	(defun is-between-cond (a b c)
		(cond
			((> a b) (< a c))
			((< a b) (> a c))))

	;; and/or
	(defun is-between (a b c)
		(or (and (> a b) (< a c)) (and (> a c) (< a b))))
\end{lstlisting}

\section{Задание №9}

Переписать функцию how-alike, приведенную в лекции и использующую COND, используя только конструкции IF, AND/OR.

\begin{lstlisting}[language=Lisp]
	;; cond
	(defun how_alike (x y)
		(cond ((or (= x y) (equal x y)) 'the_same)
			((and (oddp x) (oddp y)) 'both_odd)
			((and (evenp x) (evenp y)) 'both_even)
			(t 'difference) ) )
	
	;; if
	(defun how-alike-if (x y)
		(if (or (= x y) (equal x y)) 'the_same
		(if (and (oddp x) (oddp y)) 'both_odd
		(if (and (evenp x) (evenp y)) 'both_even 'difference))))

	;; and/or
	(defun how-alike-and-or (x y)
		(or (and (or (= x y) (equal x y)) 'the_same)
			(and (and (oddp x) (oddp y)) 'both_odd)
			(and (and (evenp x) (evenp y)) 'both_even)
			'difference))
\end{lstlisting}
