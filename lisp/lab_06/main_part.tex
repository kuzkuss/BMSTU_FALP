\chapter{Практические задания}

Используя рекурсию:

\section{Задание №1}

Написать хвостовую рекурсивную функцию my-reverse, которая развернет верхний уровень своего списка-аргумента lst.

\begin{lstlisting}[language=Lisp]
	(defun move-to (lst res)
		(cond ((null lst) res)
			(T (move-to (cdr lst) (cons (car lst) res)))))

	(defun my-reverse (lst)
		(move-to lst ()))
\end{lstlisting}

\section{Задание №2}

Написать функцию, которая возвращает первый элемент списка - аргумента, который сам является непустым списком.

\begin{lstlisting}[language=Lisp]
	 (defun get-first-list-el (lst)
		(cond ((null lst) Nil)
			((and (car lst) (listp (car lst))) (caar lst))
			(T (get-first-list-el (cdr lst)))))
\end{lstlisting}

\section{Задание №3}

Напишите рекурсивную функцию, которая умножает на заданное число-аргумент все
числа из заданного списка-аргумента, когда

a) все элементы списка --- числа,

\begin{lstlisting}[language=Lisp]
	(defun mul-nums-res (lst num res)
		(cond ((null lst) res)
			(T (mul-nums-res (cdr lst) num (cons (* (car lst) num) res)))))

	(defun mul-nums (lst num)
		(mul-nums-res lst num ()))
\end{lstlisting}

6) элементы списка --- любые объекты.

\begin{lstlisting}[language=Lisp]
	(defun mn (lst num res)
		(cond ((null lst) res)
			((numberp (car lst))
				(mn (cdr lst) num (cons (* (car lst) num) res)))
			((atom (car lst)) (mn (cdr lst) num (cons (car lst) res)))
			(T (mn (cdr lst) num (cons (mn (car lst) num ()) res)))))

	(defun mul-num (lst num)
		(mn lst num ()))
\end{lstlisting}

\section{Задание №4}

Напишите функцию, select-between, которая из списка-аргумента, содержащего только
числа, выбирает только те, которые расположены между двумя указанными границамиаргументами и возвращает их в виде списка (упорядоченного по возрастанию списка чисел
(+ 2 балла)).

\begin{lstlisting}[language=Lisp]
	(defun sb (lst begin end res)
		(cond ((null lst) res)
			((or (< begin (car lst) end) (< end (car lst) begin))
				(sb (cdr lst) begin end (cons (car lst) res)))
			(T (sb (cdr lst) begin end res))))

	(defun select-between (lst begin end)
		(sort (sb lst begin end ()) `<))
\end{lstlisting}

\section{Задание №5}

Написать рекурсивную версию (с именем rec-add) вычисления суммы чисел заданного
списка:

а) одноуровнего смешанного,

\begin{lstlisting}[language=Lisp]
	(defun ra (lst res)
		(cond ((null lst) res)
			((numberp (car lst)) (ra (cdr lst) (+ (car lst) res)))
			(T (ra (cdr lst) res))))

	(defun rec-add (lst)
		(ra lst 0))
\end{lstlisting}

б) структурированного.

\begin{lstlisting}[language=Lisp]
	(defun ra (lst res)
		(cond ((null lst) res)
			((numberp (car lst)) (ra (cdr lst) (+ (car lst) res)))
			((atom (car lst)) (ra (cdr lst) res))
			(T (ra (cdr lst) (ra (car lst) res)))))

	(defun rec-add (lst)
		(ra lst 0))
\end{lstlisting}
